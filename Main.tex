% !TEX encoding = UTF-8 Unicode
\documentclass[10pt,twoside]{article}
\usepackage[utf8]{inputenc}
\usepackage[left=2cm,right=2cm,top=2cm,bottom=2cm,bindingoffset=0cm]{geometry}
\usepackage[english,russian]{babel}
\usepackage{amsmath,amssymb,amsthm}
\usepackage{hyperref,xcolor}
\usepackage{indentfirst}
\hypersetup{pdfstartview=FitH,  linkcolor=linkcolor,urlcolor=urlcolor, colorlinks=true}
\usepackage{listings}
\usepackage{color}

\definecolor{dkgreen}{rgb}{0,0.6,0}
\definecolor{gray}{rgb}{0.5,0.5,0.5}
\definecolor{mauve}{rgb}{0.58,0,0.82}

\lstset{frame=tb,
  language=Scala,
  aboveskip=3mm,
  belowskip=3mm,
  showstringspaces=false,
  columns=flexible,
  basicstyle={\small\ttfamily},
  numbers=none,
  numberstyle=\tiny\color{gray},
  keywordstyle=\color{blue},
  commentstyle=\color{dkgreen},
  stringstyle=\color{mauve},
  breaklines=true,
  breakatwhitespace=true,
  tabsize=3
}

\sloppy

\theoremstyle{plain}
\newtheorem{thm}{Теорема}
\newtheorem{lemma}{Лемма}
\newtheorem{corol}{Следствие}
\newtheorem{prop}{Предложение}
\newtheorem{ass}{Утверждение}
\theoremstyle{definition}
\newtheorem{defi}{Определение}
\newtheorem*{remark}{Замечание}
\newtheorem*{example}{Пример}
\newtheorem{problem}{Задача}
\newtheorem{question}{Вопросы}

\definecolor{linkcolor}{HTML}{0000FF} % цвет ссылок
\definecolor{urlcolor}{HTML}{0000FF} % цвет гиперссылок

\title{Введение в функциональное програмирование на Scala}
\author{Автор \href{http://vk.com/id11423440}{Васюхин А. С.}}

\begin{document}
\maketitle
\tableofcontents

\section{Зачем нужно писать функционально}
\subsection{Введение}
Идея функционального програмирования состоит в том, чтобы писать программы используя только \textit{чистые функции}.\\
\begin{defi}
\textbf{(Нестрогое определение чистой функции)} Функция наывается \textit{чистой} если она не проиводит никаких \textit{побочных действий} в процессе своего выполнения.\\
Побочными действиями могут быть:
\begin{itemize}
\item Модификация переменной;
\item Установка значения поля у объекта;
\item Выброс исключения или ошибки;
\item Вывод или чтение из консоли;
\item Вывод или чтение из файла;
\item Отрисовка чего-либо на экране.
\end{itemize} 
\end{defi}
\subsection{Мотивационный пример}
\begin{lstlisting}
class Cafe {
  def buyCoffee(cc: CreditCard, p: Payments): Coffee = {
    val cup = new Coffee()
    p.charge(cc, cup.price)
    cup
  }
}
\end{lstlisting}

Например, подобная строка 

\texttt{def coalesce(charges: List[Charge]): List[Charge] = charges.groupBy(\_.cc).values.map(\_.reduce(\_ combine \_ )).toList}



\end{document}
